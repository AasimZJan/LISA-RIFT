% overview.tex
% RICHARD NOTE:
%   Original version:
%      Planning/ShortProjectEvaluations/2013-08-MeBradyEtAl-MarginalizedPE/overview.tex
\documentclass[twocolumn,prd,nofootinbib]{revtex4}
\newcommand\ForRichardOnly[1]{#1}
%\newcommand\ForRichardOnly[1]{}
\usepackage{verbatim}
\usepackage{color}     % color text
\usepackage{framed}
\definecolor{shadecolor}{gray}{0.95}
\usepackage{amsmath}
\usepackage{graphicx}  % extend graphics
\usepackage{tabularx}
\usepackage{wrapfig}   % wrap text around figures, if desired
\usepackage{hyperref}

% Standard LAL, LALAPPs links
\newcommand\LALDocsMain{\href{http://www.lsc-group.phys.uwm.edu/lal/slug/nightly/doc/lsd-nightly.pdf}{LALDocs}}
%\newcommand\LALappsDocsMain{\href{http://www.lsc-group.phys.uwm.edu/~saikat/doc/lalapps.pdf}{LALapps docs}}
\newcommand\LALappsDocsMain{\href{http://www.lsc-group.phys.uwm.edu/lal/slug/nightly/doc/lalapps-nightly.pdf}{LALapps docs}}
\newcommand\LALDocsSource[1]{\href{http://www.lsc-group.phys.uwm.edu/cgit/lalsuite/tree/#1}{#1}}
\newcommand\LALDocsSourceInline[2]{\href{http://www.lsc-group.phys.uwm.edu/cgit/lalsuite/tree/#1}{{\tt #2}}}

% BONUS TOOLS

\newcommand\editremark[1]{{\color{red} #1}}

\newcommand\unit[1]{\, {\rm #1}}
\newcommand\aap{A\&A}
\newcommand\apss{APSS}
\newcommand\aaps{AAPS}
\newcommand\apjs{ApJ S}
\newcommand\aj{AJ}
\newcommand\apjl{ApJL}
\newcommand\mnras{MNRAS}
\newcommand\pasp{PASP}
\newcommand\araa{ARA\&A}
\newcommand\physrep{Phys. Rep.}
\newcommand\aapr{AAPR}
\newcommand\nar{NAR}  % New astronomy reviews

\newcommand\Y[1]{{{}_{#1}Y}}
% GR tools
\newcommand\prederiv[2]{{}^{(#1)}#2}
\newcommand\dualBack{*}
\newcommand\dualForward{\star}
\newcommand\avL{\left< {\cal L}_{(a} {\cal L}_{b)} \right>}
\newcommand\WeylScalar{{\psi_4}}
\newcommand\WeylScalarFourier{{\tilde{\psi}_4}}
\newcommand\mc{{{\cal M}_c}}
% QM TOOLS
\newcommand\qmstate[1]{\left|#1\right \rangle}
\newcommand\qmstateKet[1]{\left\langle#1\right|}
\newcommand\qmstateproduct[2]{\left\langle#1|#2\right\rangle}
\newcommand\qmoperatorelement[3]{\left\langle#1\left|#2\right|#3\right\rangle}
\newcommand\qmoperator[1]{{\bf #1}}


\begin{document}

\title{PE by Monte Carlo}
\author{R. O'Shaughnessy}
\begin{abstract}
Notes. Main content in white.  Gray sidebars are ``footnotes'': implementation tricks and comments to be skipped on a
first read.
\end{abstract}
\maketitle
\part{Notes for group}
\nocite{gwastro-HarryFairhurst-CoherentTargetedSearch}
% Keppel http://adsabs.harvard.edu/cgi-bin/nph-data_query?bibcode=2012arXiv1208.2340K&link_type=ABSTRACT

\section{Stage 0: What are we calculating}

\subsection{The reduced likelihood}
\noindent \textbf{Gravitational wave signal}: Paramters $\lambda,\theta$ for $\lambda$ intrinsic parameters (slow) and
$\theta$ extrinsic parameters (fast).  The code provides $h(t|\lambda,\theta)$ in the Earth's barycenter or, more
usefeul, $h_{lm}$, a spin-weighted spherical harmonic decomposition relative to the emission direction $\hat{N}$
\begin{shaded}
\begin{itemize}
\item $h(t)=h_+-i h_\times$ is a complex strain unless explicitly indicated otherwise.   Capital letters $H$ denote
  quantities in a particular detector's data stream.  The hatted quantity $\hat{H}_k$ is the measured strain data in the
  $k$th detector.
\item All operations respect polarization symmetry and time translation.  Under a rotation by $\psi$ around the
  propagation direction, the waveform transforms as
\[
h' = h \exp(-2i \psi)
\]
\item The source propagation direction away from the source is $\hat{n}(\theta_{JN},\phi_{JN})$ where
  $\theta_{JN},\phi_{JN}$ are the angles of the propagation direction relative to the total angular momentum vector (at
  some time) $J$; the projection of $J$
  onto the plane of the sky is $\psi_J$.  The spin-weighted
  spherical harmonic decomposition is 
\begin{eqnarray}
\label{eq:def:hSpinWeightEmissionDirection}
h(t|\lambda,\theta) = \sum_{lm} h_{lm}(t|\lambda) e^{-2i\psi_J}\Y{-2}_{lm}(\theta_{JN}\phi_{JN})
\end{eqnarray}
Note $h_{lm}$ depends only on intrinsic parameters.  (Implicitly, $h_{lm}$ also depends on our choice of reference
frequency, typically 100 Hz.).
\item The sky location is $\hat{n}=\hat{n}(\delta,\alpha)$ (i.e., RA and DEC angles).  we do not have to do sky location
  conversions ourselves.
\item Fourier sign convention is as follows, consistent with \cite{gwastro-mergers-nr-Alignment-ROS-Polarization}
\begin{eqnarray}
h(t) = \int \frac{d \omega}{2\pi} \; e^{-i\omega t} h(\omega) \\
h(2\pi f) = \int dt e^{i 2\pi f t} h(t)
\end{eqnarray}
In the common case that  $h = Ae^{-i\Phi(t)}$ with $\Phi$ increasing monotonically, the fourier transform $\tilde{h}(f)$ will be dominated by
positive-frequency components
\end{itemize}
In practice $h(t)$ is timesampled at some fixed rate for its entire duration.  The signal duration is dominated by
low-frequency content.
\end{shaded}


\noindent \textbf{Response of each detector}: Each detector has a response function which returns some $H_k(t)$, the
``strain'' function of that detector.  No matter how complicated the response, it must satisfy time-translation symmetry and spin-weight-$-2$ symmetry.
 \emph{Generic} detector response must account for arbitrary frequency
content at each time and requires slow calculations to carefully propagate signals with $f\simeq f_{FSR}$.  

In the
\emph{long wavelength limit} the response can be approximated using some $F_+,F_\times(t)$:
\begin{align}
H_k(t) &=F_{+,k}(t) h_+(t-\vec{x}_k(t)\cdot \hat{k}) + F_\times(t) h_\times(t-\vec{x}_k(t)\cdot \hat{k}) \\
 &= \text{Re}(F_++i F_\times)_k h(t-\vec{x}_k(t)\cdot \hat{k}) \\
 &=  \frac{F h }{2} + \frac{F^*h^*}{2} \equiv (Fh  + F^*{\cal I} h)/2
\end{align}
where $-\hat{k}$ points toward the sky location, \emph{opposite} to the direction of propagation $\hat{k}$; where $F=F_++i F_\times$;
where $F_{+,\times}$ are calculated as $F_+(t)=e_+^{ab}d_{ab}(t)/2$ and follow by contracting a unit tensor with the time-dependent geometry tensor of each
interferometer; where self-evident dependence on $t$ and $t-\hat{k}\cdot \vec{x}_k$ has been suppressed; and
where ${\cal I}$ is the (complex-antilinear!) complex conjugation operation in time.   
\begin{shaded}
\noindent \textbf{A ``complex conjugate''}: What is  ${\cal I}$? I want to take fourier transforms of the complex
conjugate of a  complex function of time
unambiguously.  So ${\cal I}$ is the ``complex conjugate in time'' operation:
\begin{align}
{\cal I}\tilde{h}(\omega) &\equiv \int e^{-i\omega t} h(t)^* dt  = [\tilde{h}(-\omega)]^* \\
{\cal I} p h &= p^* {\cal I} h \\
(a,{\cal I} b) & = (b,{\cal I}a) = 2 \int_{-\infty}^{\infty} df \frac{[\tilde{a}(f)]^*[\tilde{b}(-f)]^*}{S_h(|f|)}\\
 &= ({\cal I} a,b)^*
\end{align}
which trivially satisfies ${\cal I}^2=1$.  Remember ${\cal I}$ is \emph{not} linear, as ${\cal I} i h = -i {\cal I} h$
and hence ${\cal I}F=F^*{\cal I}$.  
Because of how it prefers time, however, it will \emph{not} conjugate  $i\omega$ (i.e., time translations):  if $h'(t)=h(t-\Delta t)$, then 
\begin{eqnarray}
\tilde{h}' = e^{-i\omega \Delta t}\tilde{h} \qquad {\cal I}\tilde{h}' = e^{-i\omega \Delta t} {\cal I}\tilde{h}'
\end{eqnarray}


\noindent \textbf{Complex inner products}: I want to use a hilbert space, allowing complex arguments.  My preferred inner product is
therefore complex-valued; see \cite{gwastro-mergers-HeeSuk-FisherMatrixWithAmplitudeCorrections}. 
\end{shaded}

\begin{shaded}
\noindent \textbf{Evaluating the response functions}
The beampattern operations $F_{+,\times}$ are applied in the time domain. The translation to each detector position is
applied using a set of short fourier transforms.  The cost to construct $H_k$ from $h$ is therefore  $\simeq N_{samp}\ln
N_{samp}$ for $N_{samp} = \ln
T/\Delta t$.

For a sufficiently short-duration source, both the beampattern functions and detector positions can be approximated by
constants.  Strictly, this limit does not hold for binary neutron stars.  As a first approximation, however, we can
probably assume  the fourier transform $\tilde{H}_k[h]$ can be re-expressed as 
\begin{eqnarray}
\tilde{H}_k = \frac{e^{i\omega \hat{k}\cdot \vec{x}_k}}{2}\left[ 
   F_k  \tilde{h} + F_k^* {\cal I} \tilde{h} 
 \right]
\end{eqnarray}
[Physically, equation reflects a common time translation applied to two basis signals appearing in  $\text{Re} F h$.]
\end{shaded}


\noindent \textbf{Likelihood}: Each detector has a noise power spectrum $S_{h,k}$, assumed known and
stationary, defining an inner product $\qmstateproduct{a}{b}_k \equiv 2 \int_{-\infty}^\infty  a^*(f)b(f)/S_{h,k}(f)$ on
complex-valued functions $a,b$ of time.  Each detector has strain data $\hat{H}_k(t)$.  We evaluate the following likelihood ratio for each element
\begin{eqnarray}
-2\ln L_k &= \qmstateproduct{H_k-\hat{H}_k}{H_k-\hat{H}_k}_k - \qmstateproduct{\hat{H}_k}{\hat{H}_k}_k \\
  &= \qmstateproduct{H_k}{H_k}_k - 2 \text{Re} \qmstateproduct{H_k}{\hat{H}_k}_k 
\end{eqnarray}
The total likelihood is the product of individual detectors' likelihoods:
\begin{eqnarray}
\ln L &\equiv \sum_k \ln L_k  = \ln L_{\rm model} + \ln L_{\rm data} \\
\ln L_{\rm model} &\equiv -\frac{1}{2} \sum_k \qmstateproduct{H_k}{H_k}_k \\
\ln L_{\rm data} &\equiv  \sum_k \qmstateproduct{H_k}{\hat{H}_k}_k 
\end{eqnarray}
\begin{shaded}
\textbf{Efficiency of terms?}: As described below, the two terms in the log likelihood
\begin{itemize}
\item \emph{Model-only term}: This time-translation-independent $L_{\rm model}$ can be efficienly calculated from $\tilde{h}$ in the
  barycenter, for any sky location [Eq. \editremark{X}], from an O(N) operation.  To get $\tilde{h}$ in the barycenter for
  an arbitrary \emph{emission} direction, we need to sum over $\tilde{h}_{lm}(f)$ via
  Eq. (\ref{eq:def:hSpinWeightEmissionDirection})
\item \emph{Data term}: The other term can be efficiently computed for all times, sky locations, and emission directions
  by archiving $\qmstateproduct{h_{lm}\exp(-i\omega t)}{\hat{H}_k}$ timeseries [Eq. (\ref{eq:IndividualDetectorLikelihoodTimeseries:ViaSpinWeightBasis})].
\end{itemize}
\end{shaded}

\begin{widetext}
\begin{shaded}
\noindent \textbf{Reorganize?}
Using the notation in  \citet{gwastro-mergers-HeeSuk-CompareToPE-Aligned}, we define 
\begin{eqnarray}
-2 L &\equiv \rho^2 - 2 \sum_k \qmstateproduct{H_k}{\hat{H}_k}
\end{eqnarray}
The first term is time-translation invariant and ``easy'' to evaluate ``once and for all'' -- see later tricks with
$h_{lm}$.   The second term can be resummed by substituting the value of $H_k$ in terms of translations, beampatterns,
and $h(t)$ at the barycenter.  So...


\noindent \textbf{Coherent likelihood for a stationary network}: Assume $\vec{x}_k$ and $F_k$ are  stationary, so
$\tilde{H}_k(f)$ is easily evaluated and substituted: 
%\editremark{fix definitions: be consistent with old notation; make sure I include both terms}
\begin{align}
\label{eq:def:BarycenteredInnerProductWithData}
\sum_k \qmstateproduct{H_k}{\hat{H}_k}_k 
& = 
 \sum_k \frac{1}{2}\left [ \qmstateproduct{ F_k h +  F_k^*{\cal I}h}{e^{i\omega \hat{k}\cdot \hat{x}_k} \hat{H}_k}_k
 \right]
= \text{Re} \qmstateproduct{h}{{\cal S}}_{\rm ref} \\
\tilde{{\cal S}}(f) &= \sum_k \hat{H}_k(f) \frac{S_{\rm ref}}{S_{k}} F_k^* e^{-i\omega \hat{k}\cdot \vec{x}_k} \equiv \sum_k {\cal S}_k\\
{\cal S}(t) &= \sum_k F_k^* H_k(t+\vec{x}_k\cdot \hat{k}) \quad \text{if all detectors identical}
\end{align}
The inner product $\qmstateproduct{.}{.}_{\rm ref}$ uses a reference PSD; exact equality holds for any choice, including
a geometric mean PSD or a single reference interferometer:
\begin{eqnarray}
S_{\rm ref}^{-1} = \frac{1}{N_{\rm det}} \sum_k S_{k}(f)^{-1} \quad ;\text{or} \quad  S_{\rm ref} = S_1
\end{eqnarray}
%
In this form, the translated detector data carries all information about the sky location; no information about the
source is used.  The data from each detector is \emph{barycentered} by first applying the \emph{inverse} time
translation operation; then the beampattern response and detector PSD, as a weight.
%

Similarly, the signal amplitude can be expressed using barycentered quantities.  Taking care to allow for different PSDs
in different detectors, we find
\begin{align}
\label{eq:CalculateRhoFromBarycenterStrain}
\rho^2& \equiv \sum_k \qmstateproduct{H_k}{H_k}_k 
 =  \frac{1}{2}\qmoperatorelement{h}{\sum_k F_k^* F_k\frac{S_{\rm ref}}{S_k}}{h}_{\rm ref} 
 + \frac{1}{4}\qmoperatorelement{h}{\sum_k F_k^* F_k^* \frac{S_{\rm ref}}{S_k}}{{\cal I} h}_{\rm ref}  
 + \frac{1}{4}\qmoperatorelement{{\cal     I}h}{\sum_k F_k F_k \frac{S_{\rm ref}}{S_k}}{h}_{\rm ref}  \\
&\equiv  \qmoperatorelement{h}{\sigma}{h}_{\rm ref}  + \frac{\text{Re}}{2} \qmoperatorelement{h}{\zeta^*}{{\cal I}
   h}_{\rm ref}
\end{align}
where the last expression implicitly defines the operators $\sigma(f|\hat{k}), \zeta(f|\hat{k})$:
\begin{eqnarray}
\sigma \equiv \frac{1}{2} \sum_k F_k^* F_k^* \frac{S_{\rm ref}(f)}{S_k(f)} 
\qquad 
\zeta \equiv \sum_k F_k F_k \frac{S_{\rm ref}(f)}{S_k(f)} 
\end{eqnarray}

\noindent \textbf{Why do we care, mark 1}: This operation [Eq. (\ref{eq:def:BarycenteredInnerProductWithData})] allows us to quickly explore any individual sky location.  And to
explore alternative sky locations (via all possible time translations of the raw data): just reconstruct ${\cal S}$ from
the raw data streams slightly differently.   And to reduce the number of FFTs we need to perform for time translation:
having constructed ${\cal S}$, it needs to be fourier transformed only once to make the network SNR timeseries.

\noindent \textbf{Why do we care, mark 2}: Suppose we tabulate $\tilde{h}_{lm}(f)$. We can efficiently generate
$\tilde{h}$ for any emission direction [Eq. (\ref{eq:def:hSpinWeightEmissionDirection})] via an order N operation
($\tilde{h}$ depends only on emission direction, not sky position),
then efficiently generate $\rho^2$ for any sky location via an O(N) operation [the inner product with these
  sky-position-dependent operations].  
\end{shaded}
\end{widetext}




\noindent \textbf{Reduced likelihood}: The reduced likelihood combines all detectors and integrates over all prior
volume:
\begin{eqnarray}
{\color{blue} L_{\rm red} = \int p(\theta) d\theta \prod_k L_k = \int p_s(\theta) d\theta \frac{p}{p_s}\prod_k L_k }
\end{eqnarray}
where $p_s$ is our sampling prior.  Specifically, if $d_{\rm max}$ is the maximum distance allowed and $T_{window}$ the
time window, the reduced integral has the form (for \textbf{Euclidean cosmology})
\begin{eqnarray}
L_{\rm red} = \int \frac{dt}{T_{\rm window}} \frac{d^2 dd d\Omega_{sky} }{V_{\rm max}} \frac{d\Omega_{\rm
    emit}}{4\pi} \frac{d\psi}{\pi} L
\end{eqnarray}




\subsection{Stage 0: Marginalize in time}
\noindent \textbf{Rationale}: For long (BNS) signals, constructing the waveform for a fixed propagation direction is expensive, because of the cost of time translation to
each site.  Let's get the most out of each sky location


\noindent \textbf{Method: FFT translation?}: The fourier transform of the inner product provides automatic time
translation on a discrete timesample grid $t_p = t_o + n\Delta t$, using a discrete FFT approximation to
\begin{align}
-2 \ln L_k(t_p)& = \qmstateproduct{H_k}{H_k}_k + 4\text{Re}\int_{-\infty}^\infty df 
  \frac{e^{-i2\pi f t_p}\tilde{H}_k(f)\tilde{\hat{H}}_k(f)}{S_{h,k}(f)} \\
-2 \ln L(t_p) & = \rho^2 -4 \text{Re}\int_{-\infty}^{\infty} df e^{-i2\pi f t_p} \sum_k\frac{\tilde{H}_k(f)\tilde{\hat{H}}_k(f)}{S_{h,k}(f)} \\
&\equiv \rho^2 - 2 \rho \hat{\rho}(t)
\end{align}
where the first term  $\qmstateproduct{H_k}{H_k}_k$ must be independent of time and where the second term defines
$\hat{\rho}(t)$, a function of time, the data, and the template.   Marginalization over event time therefore corresponds to using these sample points to evaluate
the integral:
\begin{eqnarray}
\int dt L p_t(t)  = \frac{1}{T_{window}} \int_{t_{start}}^{t_{start}+T_{window}}dt  e^{\ln L}
\end{eqnarray}

\begin{shaded}
\noindent \textbf{Quadrature}: A typical low sample rate (few kHz) is often not enough to
evaluate the integrand in the likelihood with a naive (box) quadrature.  Very high sampling rates \textbf{or interpolating the
  integrand $L_k$  and integrading accordingly}  are needed to integrate the likelihood.

We can use any intelligent quadrature method we want for the integral -- including locally interpolating the integrand
$L$ to make a locally continuous function and integrating with any quadrature desired. 

%For speed, we may want a \emph{rejection approximation}: approximate the integrand as \emph{zero} if the max log
%likelihood is less than a certain value, or the posterior is distributed widely in time.  (Not clear if this saves us
%much: we have already done the FFT.)

\noindent \textbf{${\cal S}$ and coherent data}:  If organized intelligently, only one inverse FFT is needed to
construct $\hat{\rho}(t)$ for each sky location.  [An even better scheme uses only one inverse FFT per detector for
  \emph{all} sky locations...see $h_{lm}$ later.]

%% \noindent \textbf{Massively less than unity?}: For a physical source,  the log likelihood will be massively less than
%% unity almost everywhere.   If we just want reliable parameter estimates
\end{shaded}


\subsection{Stage 1: Marginalize in time and polarization angle}

\noindent \textbf{Rationale}: As above.
%
In GW searches, time and polarization phase marginalization is done simultaneously.  At each timesample, we expected we could
 marginalize \emph{analytically} over polarization phase, with a bessel function.

That's not true.

\noindent \textbf{Challenges}: In general the detector network and source (aka data) have preferred orientations.  These
orientations need not agree.  
\begin{shaded}
\noindent \textbf{Template amplitude}: Using the $\sigma,\epsilon$ notation of \citet{CutlerFlanagan:1994}, the signal amplitude term has the form $\rho^2 = \sigma
\rho_0^2(1+\epsilon \cos 4(\psi-\psi_0))$ where $\rho_0^2$ is the amplitude of a source directly overhead a single
detector and $\sigma,\epsilon$ are derived from the beampattern functions:
\begin{eqnarray}
C_{\lambda \lambda'}&=& \sum_k \begin{bmatrix}
F_+^2 & F_+F_\times \\
F_\times F_+ & F_\times^2
\end{bmatrix}\\
&=& \sum_k [P_k : \hat{e}_\lambda(\hat{n})] [P_k : \hat{e}_{\lambda'}(\hat{n})]  \nonumber \\
& =& \sigma U\begin{bmatrix} (1+\epsilon) & 0 \\ 0 & 1-\epsilon \end{bmatrix} U^{-1} 
\end{eqnarray}
where $U$ is a unitary transformation that diagonalizes $C$.

\noindent \textbf{Response against data}: By contrast, for a fixed data set, the second term in the likelihood
necessarily transforms like $h$: purely sinusoidally in polarization angle
\begin{eqnarray}
\rho \hat{\rho} = A \cos 2\psi + B \cos 2 \psi
\end{eqnarray}
\noindent \textbf{Marginalizing?}: If this were the only term in the likelihood -- for example, for an isotropic network
--  we could marginalize exactly: \textbf{BUT THIS IS NOT TRUE}
\begin{align}
\ln L  &= [\ln L]_c \cos 2\psi + [\ln L]_s \sin 2\psi = \ln L_0 \cos 2(\psi-\psi_0)  \\
\ln L_0 &= \sqrt{[\ln L]_c^2+[\ln L]_c^2} \\
\int_0^{\pi}\frac{ d\psi}{\pi} L &= \frac{1}{\pi}I_0(\ln L_0)
\end{align}
In general $\rho$ depends on $\psi$, so we can't: both $\cos 2\phi$ and $\cos 4\psi$ appear in the exponential, unless
the network has $\epsilon=0$ (i.e., equal sensitivity to both polarizations).
\end{shaded}



\noindent \textbf{Method: Phase alone}: We know how to evaluate all the terms in the likelihood given 4 quadrature
points at $0,\pi/4, \pi/8, 3\pi/8$.  We could hardcode the integral
\begin{eqnarray}
\int_0^{\pi}\frac{ d\psi}{\pi} e^{\ln L}
\end{eqnarray}
using values at these 4 points.   Or otherwise provide a fast way to evaluate the 1d integral, given the known form of
$\rho$ and $\qmstateproduct{h}{{\cal S}}$ veruss $\psi$

\noindent \textbf{Method: Phase and time integral}: Marginalizing over phase can be performed at each $t_p$ using the
archived values  $L(t_p, \psi)$ computed by fourier transforming the signal for several $\psi$...or by other methods.
\begin{eqnarray}
\int \frac{dt}{T_{\rm win}} \int \frac{\psi}{\pi}L
\end{eqnarray}




\subsection{Stage 2: Marginalize in time, polarization angle, and distance }

\noindent \textbf{Rationale}: As above.

\noindent \textbf{Challenges}: No analytic simplifications possible here or henceforth 

\noindent \textbf{Principle}: In each $\ln L_k$  are terms that scale quadratically with $1/d$
($\qmstateproduct{H}{H}$) and terms that scale linearly with $1/d$ ($\qmstateproduct{H}{\hat{H}}$).  If we pick one
reference value, we can rescale all terms in $L$ analytically:
\begin{align}
-2 \ln L &= \rho_{\rm ref}^2 (d_{ref}/r)^2  - 2 (d_{\rm ref}/r) \hat{\rho}\rho_{\rm ref} \\
\int L &=  \frac{dt}{T_{\rm window}} \frac{d\psi}{\pi}\int_0^{r_{rm max}} \frac{r^2 dr}{d_{\rm max}^3/3} e^{\ln L}
\end{align}
Note $\rho_{\rm ref}$ will be independent of $t$ but will depend on $\psi$, while $\hat{\rho}$

\noindent \textbf{Implementation: Generic quadrature}: All elements of the quadrature can be evaluated analytically,
from the real and imaginary parts of the fourier-transformed $\qmstateproduct{h}{{\cal S}}$.  So use something accurate
and fast -- e.g., a Monte Carlo integral with some broad distribution. 


\begin{shaded}
\noindent \textbf{Compare to  Bayestar}: Larry and Leo do a 1d distance integration with GSL : \texttt{gsl integration qagp}
\end{shaded}



\subsection{Stage 3: Marginalize in time, phase, distance, and emission direction  (*)}

\noindent \textbf{Rationale}: As above

\noindent \textbf{Principle}: Express $h(t)$ and $h(f)$ using basis functions:
\begin{eqnarray}
h(t|\lambda,\theta)=\sum_{lm}h_{lm}(t)\Y{-2}_{lm}(\theta_{NJ},\phi_{NJ}) \\
h(f|\lambda,\theta)=\sum_{lm}\tilde{h}_{lm}(f)\Y{-2}_{lm}(\theta_{NJ},\phi_{NJ})
\end{eqnarray}
This identifies natural but \textbf{not   generally orthogonal} basis functions $h_{lm}$.\footnote{Particularly with higher
  harmonics and spin, orthogonality should not be assumed.}  

\noindent \textbf{Is this worth writing out?}: If we have code to generate $h(t)$ and $h(f)$ from $h_{lm}(t)$ and
$h_{lm}(f)$, do we gain anything by writing out these expansions in full?  Just have your code provide a fast cached way
to compute $h(t)$ and $h_{lm}(f)$ so we can efficiently evaluate $\rho^2$ and $\qmstateproduct{h}{{\cal S}}$?

\noindent \textbf{What could be our advantage?}: To push extreme performance, it's \emph{possible} to generate complex-valued
timeseries   \editremark{SANITY CHECK ME: make sure intelligent use of signs}
\begin{eqnarray}
{\color{blue}Q_{k,lm} \equiv \qmstateproduct{\exp(-i\omega t) h_{lm}}{\hat{H}_k}_k} \\
{\color{blue}P_{k,lm} \equiv \qmstateproduct{\exp(-i\omega t){\cal I}h_{lm}}{\hat{H}_k}_k}
\end{eqnarray}
 once and for all, at each detector, then timeshift
the resulting timeseries  to cover all sky positions as well as all orientations.  \textbf{After the
initial signal generation, fourier, comparison against data and inverse FFT, no further fourier transforms are performed. }
%
If fourier transforms dominate our computational cost, this postprocessing-only approach will be much faster.

\begin{widetext}
\noindent \emph{Longhand Implementation}: Explicit non-barycentered expressions allowing for timeshifts or sky position
shifts have the form
\begin{eqnarray}
\label{eq:IndividualDetectorLikelihoodTimeseries:ViaSpinWeightBasis}
\sum_k\qmstateproduct{e^{-i \omega t} H_k}{\hat{H}_k}_k
= \sum_k (F_k \Y{-2}_{lm})^* {\color{blue} \qmstateproduct{ h_{lm}}{e^{i\omega t}\hat{H}_k}_k}
   + (F_k \Y{-2}_{lm}) {\color{blue}\qmstateproduct{{\cal I} h_{lm}}{e^{i\omega t}\hat{H}_k}_k} \\
\end{eqnarray}
The blue expressions have the form $\qmstateproduct{h_{lm}}{\hat{H}_k}_k$.  For a nonprecessing source ${\cal I} h_{lm}
=(-1)^l h_{l,-m}$, letting us reduce by $\times 2$ the number of inverse FFTs needed; in general no symmetry holds. 

Similarly, the signal amplitude $\rho$ can be expanded.  Because $\qmstateproduct{h_{lm}}{h_{l'm'}}\ne 0$ in general,
this expansion is \textbf{not useful}: it's easier to compute $\rho^2_k(t)$ in each interferometer from the resummed
signal, once and for all, then use timeshifts to construct the network amplitude $\rho^2_k(t-\vec{x}_k\cdot \hat{k})$.

\editremark{Problem: This costs one inverse FFT for each source orientation, to construct $\rho_k(t)$.  In special cases
where I *do* have orthogonality, I should use it}
\end{widetext}

\begin{shaded}
\noindent \textbf{Basis signals?}: Several methods exist to identify natural basis signals in
$h$.  For generic sources, some polarization angle $\psi$ exists such that for $h'\equiv h \exp(-2i\psi)$,
$h_+'$ and  $h'_\times$ are  orthogonal\footnote{%
The matrix $B_{ss'} =\qmstateproduct{h_s}{h_{s'}}$ with $s=+,\times$ is a positive-definite symmetric matrix.  Rotating
to its eigenbasis proves the claim and identifies the polarization angle needed.}
For generic sources,  the positive- and negative-frequency parts of $h$ are also trivially orthogonal and  correspond to left- and
right-handed radiation.\footnote{ in simple cases,  these two parts may be proportional to complex-conjugates of one
another but that breaks down in general.  }
In general we don't have ``cosine'' and ``sine'' chirps, particularly with higher harmonics and spin.

This situation should be contrasted with the assumptions implicit in the F statistic
\cite{gwastro-HarryFairhurst-CoherentTargetedSearch}, which de facto makes the cosine/sine chirp assumption; BCV
\cite{BCV:PTF}; et cetera. 
\end{shaded}


%% \begin{shaded}
%% \noindent \textbf{Barycentering and $\rho^2$}:  We described how to barycenter the data in
%% Eq. (\ref{eq:def:BarycenteredInnerProductWithData}).  Similar expressions let us barycenter the signal amplitude
%% calculation: 
%% \begin{eqnarray}
%% \rho^2 \equiv \sum_k \qmstateproduct{H_k}{H_k}_k
%% \end{eqnarray}
%% \begin{align}
%% \sum_k \qmstateproduct{H_k}{H_k'} &
%% =
%%  \sum_k F_k^* F_k' \qmoperatorelement{h}{ e^{-i\omega \hat{x}_k\cdot(\hat{n}'-\hat{n})}}{h'}/4 
%% \nonumber \\
%% &  + \sum_k F_k (F_k')^*  \qmoperatorelement{{\cal I}h'}{ e^{+i\omega
%%     \hat{x}_k\cdot(\hat{n}'-\hat{n})}}{{\cal I}h}/4 \nonumber \\
%% &+ \frac{1}{4} 
%% \qmoperatorelement{h}{\sum_k F_k^* (F_k')^* e^{-2i\omega \hat{n}\cdot x_k} }{{\cal I}h}
%% \nonumber \\
%% &+  \frac{1}{4}
%% \qmoperatorelement{{\cal I}h}{ \sum_k F_k F_k'  e^{+2i\omega \hat{n}\cdot x_k}}{h}
%% \end{align}
%% Motivated by the form of this expression and by  \citet{CutlerFlanagan:1994}, we define the operators $\sigma(\hat{n},\hat{n'})$ and
%% $\zeta(\hat{n},\hat{n'})$ by 
%% \begin{align}
%% \sigma(\hat{n},\hat{n})&\equiv   \sum_k  F_k^* F_k'  e^{-i\omega \hat{x}_k\cdot(\hat{n}'-\hat{n})}/2  \\
%% \zeta(\hat{n},\hat{n}')& \equiv \sum_k F_k F_k' e^{i\omega (\hat{n} + \hat{n}')\cdot x_k}
%% \end{align}
%% The operator $\sigma$ is a function (number) if $\hat{k}=\hat{k}'$.  
%% %
%% In terms of these operators, we can re-express this inner product as 
%% \begin{align}
%% \sum_k \qmstateproduct{H_k}{H_k'} & = \frac{\qmoperatorelement{h}{\sigma}{h'} +\qmoperatorelement{h}{\sigma}{h}^* }{2}
%% \nonumber \\ &
%% +  \frac{\qmoperatorelement{h}{\zeta^{\dag}}{{\cal I}h'} +\qmoperatorelement{h}{\zeta^\dag}{{\cal I}h}^* }{4}
%% \nonumber \\
%% & =  \text{Re} \left[ \qmoperatorelement{h}{\sigma}{h'} + \frac{1}{2}  \qmoperatorelement{h}{\zeta^\dag}{{\cal I} h'} \right]
%% \end{align} 
%% % POINT: Relation to ``coherent network amplitude''
%% The form of this expansion suggests that the following quantity can reasonably be called the ``complex network
%% response'' to a signal $h$ in the Earth's barycenter:
%% \begin{eqnarray}
%% {\cal S} h \equiv [\sigma  + \frac{1}{2} \zeta^{\dag} {\cal I} ] h
%% \end{eqnarray}
%% \end{shaded}


\subsection{Stage 4: Marginalize in time, phase, distance,  emission direction, and sky location  (*)}

\noindent \textbf{Rationale}: As above

\noindent \textbf{Concept}:  Using a very small number of archived $Q_{k,lm},P_{k,lm}$ timeseries points, we can
reconstruct $L_{\rm data}$ very quickly for any extrinsic parameters (much less than $O(N)$).  

For each emission direction and/or sky position, I think we still need an O(N) operation to reconstruct $h$ and to
evaluate $\rho^2$.  But $\rho^2$ is time-translation invariant and scales trivially with distance (and polarization, if
we make both basis signals).



\begin{widetext}
\subsection{Stage 5: Marginalize over all extrinsic }

\noindent \textbf{Fast extrinsic evaluation}: Assume we have $\tilde{h}_{lm}(f|\lambda)$ (signal harmonics);
$P_{k,lm}(t|\lambda),Q_{k,lm}(t|\lambda)$ (data versus signal for each detector and harmonic).  \textbf{We only need P,Q
  over a short window time.}  These calculations
require a fixed distance scale; we adopt $d_{\rm ref}=100\unit{Mpc}$ by default.  We assume time-dependent quantities
have been interpolated, to allow high resolution.  Using these quantities, we compute
$L_{\rm data}$ and $L_{\rm model}$ for any event time via
\begin{itemize}
\item \emph{Data term}:  The scheme for computing $L_{\rm data}$ was outlined above:
\begin{align}
\ln L_{\rm data}(\lambda|t,\hat{n},\hat{k},\psi_J,d) &= \sum_k\qmstateproduct{H_k}{\hat{H}_k}_k \nonumber \\
& =
{\color{blue} (d_{\rm ref}/d) \sum_k \sum_{lm}(F_k(-\hat{k}) e^{-2\psi_J} \Y{-2}_{lm}(\hat{n}))^* Q_{k,lm}(t-\hat{k}\cdot x_k)
   + (F_k(-\hat{k}) \Y{-2}_{lm}(\hat{n}) e^{-2\psi_J}) P_{k,lm}(t-\hat{k}\cdot x_k) 
}
\end{align}
Assuming $P,Q$ are interpolated off a small number of basis points, this expression can be evaluated ``analytically''.
\item \emph{Signal term}:  The scheme for computing $L_{\rm model}$ was outlined in
  Eq. \ref{eq:CalculateRhoFromBarycenterStrain}.     To minimize the number of re-evaluations, let's make any dependence on
  $\psi_J$ and $d$ explicit
\editremark{check coefficients and signs}
\begin{align}
\tilde{h}(f|\lambda,\hat{n},d_{\rm ref}) &\equiv \tilde{h}_{\rm ref}(f) = \sum_{lm} \tilde{h}_{lm}(f|\lambda) \Y{-2}_{lm}(\hat{n}) \\
\tilde{h}(f|\lambda,\hat{n},\psi_J,d_{\rm ref})&= e^{-2i\psi_J} \tilde{h}_{\rm ref}(f) \\
\tilde{H}_k(f)_{\text{no time shift}} &= \frac{(d_{\rm ref}/d)}{2}[F_k h + F^*_k {\cal I} h]
   = \frac{1}{2}\left[
   F_k e^{-2i\psi_J} \tilde{h}_{\rm ref} +F_k^* e^{2i\psi_J}{\cal I}\tilde{h}_{\rm ref}
  \right] \\
\ln L_{\rm model}(\lambda|\hat{n},\hat{k},\psi_J,d) &= -\frac{1}{2} \sum_k\qmstateproduct{H_k}{H_k}_k 
 \\
&= -\frac{(d_{\rm ref}/d)^2}{2}
\left[ \frac{1}{2} [ \sum_k \qmstateproduct{h_{\rm    ref}}{h_{\rm ref}}_k F_k F_k ]
 +\frac{1}{2} e^{4i\psi_J} F_k^* F_k^* \qmstateproduct{h_{\rm ref}}{{\cal I}h_{\rm ref}}_k
 + \frac{1}{2} e^{-4i\psi_J} F_k F_k \qmstateproduct{{\cal I} h_{\rm ref}}{h_{\rm ref}}_k 
\right] \\
&= 
{\color{blue}
-\frac{(d_{\rm ref}/d)^2}{2}
\left[ \frac{1}{2} [ \sum_k \qmstateproduct{h_{\rm    ref}}{h_{\rm ref}}_k F_k F_k ]
 + \text{Re} e^{-4i\psi_J} F_k F_k \qmstateproduct{{\cal I} h_{\rm ref}}{h_{\rm ref}}_k 
\right]
}
\end{align}
The last expression requires $2N_{\rm det}$ inner products to evaluate.  But it can be implemented as just
one $O(N)$ loop over all frequency bins.

\end{itemize}

\noindent \textbf{Marginalization scheme}: We want to evaluate $L_{\rm red} \equiv \int p(\theta)L(\lambda,\theta)$:

\begin{itemize}
\item \emph{Easy extrinsic first}: Marginalize over $t,d$ and perhaps $\psi_J$ first: these require no $O(N)$
  operations.  For example, in the first case $L_{\rm model}=-\rho^2/2$ scales trivially as $1/d^2$.

I recommend doing this 1d or 2d integral with raw quadrature -- particularly the 1d time case, which requires some care.

\begin{shaded}
Since sky location and emission direction are fixed, we can preprocess and scale $Q_k$ and $P_k$ by their fixed
prefactors, generating short fiducial timeseries (only needed over the window time)
\begin{eqnarray}
\bar{Q}(t) \equiv  \sum_{k,lm}(F_k(-\hat{k}) \Y{-2}_{lm}(\hat{n}))^* Q_{k,lm}(t-\hat{k}\cdot x_k) \\
\bar{P}(t)\equiv   \sum_{klm} (F_k(-\hat{k}) \Y{-2}_{lm}(\hat{n}) ) P_{k,lm}(t-\hat{k}\cdot x_k)  \\
\ln L_{\rm data}(\lambda|t,\hat{n},\hat{k},\psi_J,d)  = (d_{\rm ref}/d)[e^{+i2\psi_J} \bar{Q}(t)  + e^{+i2\psi_J}
  \bar{Q}(t)] 
\end{eqnarray}
The computation of $\rho^2$ (aka $-2\ln L_{\rm model}$) is $2+2N_{\rm det}$ $O(N)$ operations, assuming we store
$\tilde{h}_{lm}(f)$: compute $\tilde{h}_{\rm ref}$ and ${\cal I} \tilde{h}_{\rm ref}$; then compute the two inner
products at each detector.    This expression has the form
\begin{eqnarray}
\ln L_{\rm model}  = -\frac{(d_{\rm ref}/d)^2}{2}\left[ A  + B \cos 4(\psi-\psi_0) \right] \\
B = |\sum_k F_k F_k \qmstateproduct{{\cal I}h_{\rm ref}}{h_{\rm ref}}_k|
\end{eqnarray}
The value of $\psi_0$ is unimportant and marginalized away.  

    Given these two terms, we compute an ``easy'' integral,
where every term in the integrand can be computed from a quick interpolation over a well-sampled, low-dimensional function:
\begin{eqnarray}
L_{red,EZ} \equiv \int \frac{dt}{T_{\rm window}} \frac{d\psi}{\pi} \frac{dr}{r_{\rm max}^3/3}e^{\ln  L}
\end{eqnarray}
\end{shaded}

\item \emph{Slow extrinsic second}: Emission direction and sky position require O(N) operations to recompute compute
  $\rho^2$, via recomputing $\tilde{h}$ from $\tilde{h}_{lm}$ or from evaluating an inner product of the form
  $\qmoperatorelement{h}{U}{h}$.    

An integral over this 4d space can be done by monte carlo -- preferably using an intelligent sampling distribution.

\end{itemize}



\end{widetext}
\subsection{Discussion: what could we do for speed? (*)}

* minimize the number of FFT evaluations -- for example, bootstrap the translation operation to each detector

* clever likelihood evaluation (e.g., chunk the likeihood into time-frequency bins, and taper between bins? Good idea,
LOTS of infrastructure: basically wavelet-basis idea.  Or just use \textbf{two} time-frequency bins: 64 sec close to the
source and anything before 64 sec at a lower sample rate (256 Hz?).  Potentially dangerous for higher harmonics)

* Reject extremely low likelihoods early on: if the log likelihood is very close to zero at the FFT stage, don't bother


\section{Stage 1: Prototype code}

\subsection{Routines needed}

\noindent \textbf{Ldata and Lsignal}: 

\noindent \textbf{Monte carlo integrator}

* Monte carlo integrator with convergence test (i.e., threshold on $L_{\rm reduced}$ and error tolerance target,
absolute and relative)

** ideally ability to multithread

\noindent \textbf{Sampling distribution infrastructure}:

* request bound on each parameter (e.g., could identify 'light ring normal' and 'light ring width'; 'time
window.barycenter', etc. )

* return random sample from each parameter

* 

\subsection{Code tests}





\section{Alternative infrastructure (*)}

* Construct intermediate transfer functions from complex methods paper.  Use a resummed complex quantity for each sky location


\bibliography{overviewexport}
\end{document}

