\documentclass[%twocolumn,
prd,nofootinbib]{revtex4}
%\newcommand\ForRichardOnly[1]{#1}
\newcommand\ForRichardOnly[1]{}
\usepackage{verbatim}
\usepackage{color}     % color text
\usepackage{framed}
\definecolor{shadecolor}{gray}{0.95}
\usepackage{amsmath}
\usepackage{graphicx}  % extend graphics
\usepackage{tabularx}
\usepackage{wrapfig}   % wrap text around figures, if desired
\usepackage{hyperref}


\begin{document}

\title{Rapid, embarassingly parallel parameter estimation of gravitational waves from compact binary coalescences}
\author{P. Brady}
\email{patrick@gravity.phys.uwm.edu}
\affiliation{Center for Gravitation and Cosmology, University of Wisconsin-Milwaukee, Milwaukee, WI 53201, USA }
\author{E. Ochsner}
\email{evano@gravity.phys.uwm.edu}
\affiliation{Center for Gravitation and Cosmology, University of Wisconsin-Milwaukee, Milwaukee, WI 53201, USA }
\author{R. O'Shaughnessy}
\email{oshaughn@gravity.phys.uwm.edu}
\affiliation{Center for Gravitation and Cosmology, University of Wisconsin-Milwaukee, Milwaukee, WI 53201, USA }
\author{C. Pankow}
\email{pankow@gravity.phys.uwm.edu}
\affiliation{Center for Gravitation and Cosmology, University of Wisconsin-Milwaukee, Milwaukee, WI 53201, USA }

\begin{abstract}
We present a method for rapid, parallelizable parameter estimation  of GWs from CBCs.
\end{abstract}
\maketitle

\section{Outline}



The first level of bullets are proposed sections. 
The second level of bullets (except in Executive Summary) are proposed subsections.
The third level of bullets are points to make in each subsection.

\begin{itemize}
\item Executive Summary
	\begin{itemize}
	\item Goal: Rapid parameter estimation of CBC GW signals -- target is a few minutes!
	\item Trick \# 1: use mode decomposition and compute $( h_{\ell m} | d )$ once for each mass pair.
		Extremely fast likelihood evaluation as extrinsic parameters are varied.
	\item Trick \# 2: Abandon Markov chain/ nested sampling. They have nice convergence in high dimensions,
		but are serial. Instead, use brute force grid and Monte Carlo technique. Convergence worse in a sense,
		but embarassingly parallel and very rapid evaluations thanks to trick \# 1.
	\item Essentially same cost regardless of waveform evaluation speed. Therefore can use very long signals 
		and/or expensive models like EOB. 
	\end{itemize}

\item Methods
	\begin{itemize}
	\item Likelihood evaluation
		\begin{itemize}
		\item Start with expression for ${\cal L}$. 
		\item Show steps to get in terms of $( h_{\ell m} | d )$.
		\item Note how all extrinsic parameters enter in $Y_{\ell m}$'s and $F_+$, $F_\times$, and thus we can
			evaluate likelihood cheaply for any extrinsic parameters if we fix the intrinsic ones.
		\end{itemize}
		
		
	\item Integration over extrinsic parameters
		\begin{itemize}
		\item Do a basic Monte Carlo integral over  extrinsic parameters, at fixed intrinsic parameters
		\item Describe any fancy pants adaption, use of skymaps, etc. 
		\end{itemize}
	\item Time marginalization  \textbf{subsection of extrinsic} 
		\begin{itemize}
		\item Inverse FFT trick gives $(h_{\ell m}(t_c) | d)$ for all values of $t_c$ at once.
		\item We just sum over a reasonably sized window $\sim 10$ ms. 
		\end{itemize}

	\item Covering the intrinsic parameters
		\begin{itemize}
		\item Describe effective Fisher approach to laying out mass points.
		\item Point out quite flexible: can do random or fixed grid, can change ellipse size, distribution inside ellipse,
			could do several ellipses centered on different points. [?]
		\item Point out difficult to go to many intrinsic dimensions. Right now we focus on 2D non-spinning, 
			but 3D or 4D is possibly feasible.
			Precession would be very tough. 
			Maybe an approximate metric or help from ROM could save the day. [?]
		\end{itemize}
		
	\end{itemize}
	
	\item Postprocessing [?]
		\begin{itemize}
		\item Do we need to say anything about collating results, making triplots, P-P plots, etc.?
		\end{itemize}

\item Results
	\begin{itemize}
	\item Show posteriors, convince reader they're similar enough to usual Bayesian PE to be trusted.
	\item P-P plots. Show our results are self-consistent.
	\item Brief (not more than 1-2 paragraphs, 1 simple plot/table) results confirming 
		that our method scales favorably with $f_{\rm min}$ and/or EOB. 
	\end{itemize}

\item Conclusions -- self-explanatory.

\item Appendices -- Put detailed technical asides here.

\end{itemize}

Citations: Bayesian methods and monte carlo integration \cite{2011RvMP...83..943V}, including numerical recipes and Lepage; comparison
to other PE methods for LIGO
(MCMC,nested sampling) \cite{LIGO-CBC-S6-PE,2011PhRvD..83h2002D,2011PhRvD..84f2003C,gr-extensions-tests-Europeans2011,gwastro-mergers-PE-Aylott-LIGOATest,2011ApJ...739...99N,2012PhRvD..85j4045V,gw-astro-PE-Raymond,gw-astro-PE-lalinference-v1}


\section{Executive Summary}

Here's the executive summary.

\section{Methods}

Methods go here.

\subsection{Efficiently evaluating the likelihood}

\subsection{Sampling the extrinsic parameters}


\subsection{Scalable method to explore intrinsic parameters }


\section{Results}

Results go here.

\subsection{Scaling}

\subsection{Detailed investigation of one event}

Sample run results: $p(m_1,m_2)$

Comparison with Fisher and MCMC

\subsection{Ensemble of events}

pp plots


\section{Conclusions}

Conclusions go here.

\appendix

\section{Notation and definitions}


\section{Planned or incomplete investigations}

\subsection{Measures of convergence}

Measures to assess whether we're done
\begin{itemize}
\item * Integral: Classic MC error estimate (central limit); reproducibility across runs and via subsamples (e.g., chisquared)

\item * 1d posterior:  $n_{\rm eff}$; $L^2$ or KL divergence between subsamples or across runs

\item * sampling distribution: similarly
\end{itemize}




\bibliography{overviewexport}
\end{document}
