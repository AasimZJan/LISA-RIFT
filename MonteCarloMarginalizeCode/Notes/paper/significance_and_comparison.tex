\section{Significance and Comparison with related work }
\label{sec:Discussion}

% POINT: Low latency
We have demonstrated a viable strategy for low-latency parameter estimation for long-duration binary neutron star signals, using
a production environment and currently-available computing configuration.   
%
Like low-latency sky localization 
\cite{gwastro-skyloc-Sidery2013,LIGO-2013-WhitePaper-CoordinatedEMObserving}, already provided with reasonable accuracy
by approximate methods like \BS{}, low-latency parameter estimation will enable rapid electromagnetic followup and interpretation of
candidate gravitational wave events. 


% POINT: Fast, enables large-scale tests, to get background and to perform other tests
%
More broadly, by dramatically decreasing the turnaround time for each analysis and by scaling to harness all available
resources efficiently, our strategy may significantly increase size and scope of parameter estimation investigations.
%
Indeed, because our code converges \emph{even more quickly} in the absence of a signal, our approach could be applied to
systematically follow up timeslides, providing a systematic understanding of the additional detection confidence
parameter estimation could provide. 
%


% POINT: Technical reasons why our procedure will be useful for environments with limited computing, 
%   like Virgo's clusters, etc
Finally, because our implementation has bounded runtime -- one hour is the \emph{worst} case -- we know what  resources
will be needed to analyze a given NS-NS binary. Moreover, the parallel algorithm can exploit all available computing
resources, without need for communication or coordination between jobs, allowing it to operate in computing environments
with tightly constrained wallclock time. 


\subsection{Reduced-order quadrature}
Reduced-order quadrature methods provide an efficient representation of a waveform family and any inner products against
it.   Other authors have recently proposed prefiltering the data against the reduced-order basis
\cite{gw-astro-ReducedOrderQuadraturePE-TiglioEtAl2014}, achieving significant speedup.  For example, using TaylorF2
templates, \cite{gw-astro-ReducedOrderQuadraturePE-TiglioEtAl2014} claim runtimes of order 1 hour, comparable to our
end-to-end time in the high-precision configuration described above.

% POINT
Our strategy and reduced-order modeling achive a similar speedup for qualitatively similar reasons: both strategies
prefilter the data.   In our algorithm, at each mass point, the data is prefiltered  against a
set of $h_{lm}$, then efficiently reconstruct the likelihood for generic source orientations and distances.  
%
By integrating the likelihood at each mass point over all extrinsic parameters,  we are dominated by extrinsic-parameter
sampling and  hence not limited by waveform generation.

% POINT: 
In the short term, reduced-order methods require further tuning and development, to calibrate their interpolation in
targeted mass regions and with specific PSDs.
Moreover, as the starting frequency is reduced, reduced-order methods do require additional basis vectors, increasing
their operation count and computational cost as $f_{\rm low}$ is reduced.
%
  By contrast,
our algorithm can be immediately applied to any noise curve and existing time-domain model that provides $h_{lm}$,  at any mass,
including EOBNRv2HM \cite{gw-astro-EOBNR-Calibrated-2009} and SEOB \cite{gw-astro-EOBspin-Tarrachini2012}.  Minimal
updates are  needed in \texttt{lalsimulation} to provide $h_{lm}$ (e.g., $h_{22}$) for most
other existing time- and frequency-domain waveform approximants.  
%
%
Finally, by construction our dominant operation count cost is  independent of the waveform's length (or number of basis
vectors).  Hence, unlike reduced order methods, our code will run in nearly same amount of time now and with full
aLIGO-scale instruments with $f_{\rm low}\simeq 10\unit{Hz}$.  


\subsection{Alternative parallelization schemes}
% POINT:
Any strategy that can compute evidence reliably over a sub-volume or hypersurface in parameter space
can be efficiently parallelized.   In the strategy described here, we parallelized via accurate, independent extrinsic marginalization.  
%
Other strategies are \textbf{probably being developed; we should ask}.
