\section{[Internal use] : Weaknesses people may ask about at LVC}

* Prior on sky is skymap

* Discrete sky grid: biases? What about high SNR? (A: just use adapted resolution; not ready yet)

* Uniform weighting

* Not discarding values after burnin

* Adaptation algorithm: why parameters chosen?

* Computational cost: is this really favorable in CPU-hours?  Are we wasting CPU hours to achieve low-latency that's not needed?

** A: Not clear our overall cost is higher...we may be more accurate than 1000 samples, the usual metric for MCMC...we need a clear comparison

* Rotation of the earth

\subsection{Runtime}

* Might be ram-limited in python below 10-ish Hz (multiple harmonics and IFOs, sampled at 16kHz for 30 min)
%	1byte/sample * 16384*30min *60s/min ~ 28 Mb 

\subsection{Applications}

* Einstein at home scaling?

* GRB project with Alex (already on table)


\section{[Internal use] Monte Carlo integration}

\textbf{Non-uniform, variance-weighted weighting}: By default we combine results with different sampling priors treating
them equally (e.g., as if they have converged to the same sampling prior) when evaluating the integral at each mass
point.  

That doesn't minimize variance and can be particularly
ineffective when one ILE run has a problem.  Really, we should use variance-reducing weighting: if $x=ax_1 +b x_2$ with $a+b=1$ and
$\left<x_1\right> =\left<x_2\right> =\left<x\right>$, we can minimize the variance of $x$ by choosing weight $a$ to
minimize $a^2 \sigma_1^2 + (1-a)^2 \sigma_2^2$, so 
\begin{eqnarray}
x = \frac{\sigma_2^2}{\sigma_1^2+\sigma_2^2} x_1 +  \frac{\sigma_1^2}{\sigma_1^2+\sigma_2^2} x_2
\end{eqnarray}
A straightforward generalization exists with arbitrary numbers of weights:
\[
x = \frac{x_k/\sigma_k^2}{\sum_q 1/\sigma_q^2}
\]
We could and should implement this scheme when recombining results at each mass point. 
%% a^2 s1^2 + b^2 s2^2 + (1 - a - b)^2 s3^3
%% Solve[{D[%, a] == 0, D[% == 0, b]}, {a, b}]

\section{[Internal use] Results: Targeted studies}

\begin{itemize}
\item Single event, 100 noise realizations, plus comparison with Fisher matrix

\item Zero-spin 2015 MDC: [\textbf{Not started}]  For completeness, we propose to reproduce the 2015 MDC,  using
  injections with exactly zero spin.

\end{itemize}

\subsection{Detailed investigation of one event}
Sample run results: Single event, with full DAG, done multiple times (for complete consistency/reproducibility)

Sample run results: extrinsic parameters (for one and several noise realizations)

Sample run results: $L_{\rm red}(\mc,eta)$ (for one and several noise realizations). Expected accuracy at each mass
point.. Translating into  $p(m_1,m_2)$ using a uniform mass prior.

Comparison with Fisher and MCMC


\begin{figure}
\caption{\label{fig:TargetedEvent:LikelihoodVersusMchirpEta}\textbf{Targeted study: Posterior distribution
    versus component masses, for several noise realizations}: : The 90\% confidence interval derived from $L_{\rm red}$.  For
comparison, the prediction from a Fisher matrix is shown as a solid black curve.
 \textbf{PLACEHOLDER/INTENT}
}
\end{figure}


\section{[Internal use] Results: Additional production-environment investigations}

\begin{itemize}
\item 2016 BNS MDC: [\textbf{Not started}]

\end{itemize}

\section{[Internal use] Incomplete or not-yet-implemented investigations}

\subsection{Measures of convergence}

Measures to assess whether we're done
\begin{itemize}
\item * Integral: Classic MC error estimate (central limit); reproducibility across runs and via subsamples (e.g., chisquared)

\item * 1d posterior:  $n_{\rm eff}$; $L^2$ or KL divergence between subsamples or across runs

\item * sampling distribution: similarly
\end{itemize}


\subsection{Semianalytic marginalization over distance}


\section{[Internal use] Future directions}

Not for distribution!

\subsection{Short GRBs}

Alex

\subsection{Adding dimensions}

Tides [Les]

Aligned spin

\subsection{Even more speed}

Because the precompute phase is \emph{much} faster than marginalizing over extrinsic parameters, we could accelerate
code performance in a number of ways, potentially enabling us to tackle higher dimensions

\noindent \textbf{Einstein at home}: Transmit the precomputed data (10ms of timeseries and a few scalars) to the
Einstein at home grid, giving each distant host (say) 20 mass points to handle.  With $10^5$ cores, that means roughly
$20\times10^5/10$ intrinsic points per day...enough to handle precessing spin faster than any other extant method.

\noindent \textbf{Interpolate $Q,U$}: Perform the precompute step several times, building up a grid for $Q$ and $U,V$,

\section{[Internal use] Monte Carlo integration}

\textbf{Non-uniform, variance-weighted weighting}: By default we combine results with different sampling priors treating
them equally (e.g., as if they have converged to the same sampling prior) when evaluating the integral at each mass
point.  

That doesn't minimize variance and can be particularly
ineffective when one ILE run has a problem.  Really, we should use variance-reducing weighting: if $x=ax_1 +b x_2$ with $a+b=1$ and
$\left<x_1\right> =\left<x_2\right> =\left<x\right>$, we can minimize the variance of $x$ by choosing weight $a$ to
minimize $a^2 \sigma_1^2 + (1-a)^2 \sigma_2^2$, so 
\begin{eqnarray}
x = \frac{\sigma_2^2}{\sigma_1^2+\sigma_2^2} x_1 +  \frac{\sigma_1^2}{\sigma_1^2+\sigma_2^2} x_2
\end{eqnarray}
A straightforward generalization exists with arbitrary numbers of weights:
\[
x = \frac{x_k/\sigma_k^2}{\sum_q 1/\sigma_q^2}
\]
We could and should implement this scheme when recombining results at each mass point. 
%% a^2 s1^2 + b^2 s2^2 + (1 - a - b)^2 s3^3
%% Solve[{D[%, a] == 0, D[% == 0, b]}, {a, b}]

\section{[Internal use] Results: Targeted studies}

\begin{itemize}
\item Single event, 100 noise realizations, plus comparison with Fisher matrix

\item Zero-spin 2015 MDC: [\textbf{Not started}]  For completeness, we propose to reproduce the 2015 MDC,  using
  injections with exactly zero spin.

\end{itemize}

\section{[Internal use] Results: Targeted studies}

\begin{itemize}
\item Single event, 100 noise realizations, plus comparison with Fisher matrix

\item Zero-spin 2015 MDC: [\textbf{Not started}]  For completeness, we propose to reproduce the 2015 MDC,  using
  injections with exactly zero spin.

\end{itemize}

\subsection{Detailed investigation of one event}
Sample run results: Single event, with full DAG, done multiple times (for complete consistency/reproducibility)

Sample run results: extrinsic parameters (for one and several noise realizations)

Sample run results: $L_{\rm red}(\mc,eta)$ (for one and several noise realizations). Expected accuracy at each mass
point.. Translating into  $p(m_1,m_2)$ using a uniform mass prior.

Comparison with Fisher and MCMC



\section{[Internal use] Results: Additional production-environment investigations}

\begin{itemize}
\item 2016 BNS MDC: [\textbf{Not started}]

\end{itemize}

\section{[Internal use] Incomplete or not-yet-implemented investigations}

\subsection{Measures of convergence}

Measures to assess whether we're done
\begin{itemize}
\item * Integral: Classic MC error estimate (central limit); reproducibility across runs and via subsamples (e.g., chisquared)

\item * 1d posterior:  $n_{\rm eff}$; $L^2$ or KL divergence between subsamples or across runs

\item * sampling distribution: similarly
\end{itemize}


\subsection{Semianalytic marginalization over distance}


\section{[Internal use] Future directions}

Not for distribution!

\subsection{Short GRBs}


\section{[Internal use] Incomplete or not-yet-implemented investigations}

\subsection{Measures of convergence}

Measures to assess whether we're done
\begin{itemize}
\item * Integral: Classic MC error estimate (central limit); reproducibility across runs and via subsamples (e.g., chisquared)
\end{itemize}

\section{[Internal use] Future directions}

Not for distribution!

\subsection{Short GRBs}

Alex

\subsection{Adding dimensions}

Tides [Les]

Aligned spin

\subsection{Even more speed}

Because the precompute phase is \emph{much} faster than marginalizing over extrinsic parameters, we could accelerate

\section{[Internal use] Patches pending or desired}

Persons/proposing supporting  a change listed as [name].

\noindent \textbf{Next push}

* Bugfix on ILE MC variance

* Uniform on sky prior with bayestar skymaps

\noindent \textbf{Other important updates}

*  [ROS]  $b_k \ne 0$ for any sky position (else complications arise re normalization).

* [ROS] Interpolation of $L$ vs masses using spokes (higher order) -- current plots are mathematica

\noindent \textbf{Minor code tweaks}

* [ROS] Larger nchunk

* [ROS] Change tempering exponent definition from $(Lp/p_s)^\beta$ to $L^\beta p/p_s$, to facilitate analysis

* [ROS] Variance weighting when recombining samples from different runs at the same mass point


\noindent \textbf{Infrastructure}

* [ROS] Change DAG (number of jobs, max iterations): current solution with skymap is overkill


* [ROS] For automated end-to-end tests, add \gstlal{} coinc generation and \BS{} skymaps to \texttt{stage\_injections}

* [ROS] Add posterior in mchirp, eta accounting for prior.
